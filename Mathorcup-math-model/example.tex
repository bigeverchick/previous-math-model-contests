% !TeX program = xelatex
% !TeX encoding = UTF-8
\documentclass{MathorCupmodeling}
\usepackage{zhlipsum,mwe}
\bianhao{123456}
\tihao{A}
\timu{MathorCup 杯 \LaTeX{} 模板写作示例}
\keyword{量子计算机;组合优化;QUBO模型}
\begin{document}
	\begin{abstract}
	请使用 \TeX Live 2019,XeLaTeX 编译,请选用支持 UTF­-8 编码的编辑器。

		使用者需要有一定的 \LaTeX{} 的使用经验({\heiti 任务期三个月以内不建议使用 \LaTeX},因此本文没有介绍基础使用),至少要会使用常用宏包的一些功能,比如参考文献,数学公式,图片使用,列表环境等等。模板已经添加了常用的宏包,无需用户再额外添加。

		本模板
		\begin{itemize}
			\item 定义了几个宏 \lstinline|\def\ee{\mathrm{e}},\def\ii{\mathrm{i}},\def\leq{\leqslant},\def\geq{\geqslant}| 方便使用;
			\item 图片应放在 \lstinline|figure| 文件夹中;
			\item 定制了 matlab 和 python 代码环境,使用方法:\lstinline|\begin{matlab} content \end{matlab}| 和 \lstinline|\begin{python} content \end{python}|;
			\item 加载了 \lstinline|cleveref| 宏包,使用方法:\lstinline|\cref{label}|。
		\end{itemize}
		其它的就是跟普通的 \lstinline|ctexart| 使用方法一样。
	
		欢迎到 \url{https://wenda.latexstudio.net/} 和 \url{https://github.com/CTeX-org/forum} 提问,注意提供 MWE,提问步骤可参考 \url{https://paste.ubuntu.com/p/wRq2mFCvWC/}。

	\end{abstract}
	\tableofcontents\newpage
	\section{问题的提出}
	\zhlipsum*[2]
	\subsection{问题的背景}
	\zhlipsum*[3]
	\subsection{问题的提出}
	\zhlipsum*[4]

	\section{问题的分析}
	\zhlipsum*[5]
	\subsection{问题的整体分析}
	\zhlipsum*[6]
	\subsection{问题一的分析}
	\zhlipsum*[7]
	\subsection{问题二的分析}
	\zhlipsum*[8]
	\subsection{问题三的分析}
	\zhlipsum*[9]

	\section{模型的假设}
	\begin{enumerate}
		\item content;
		\item content;
		\item content;
		\item content;
		\item content
	\end{enumerate}
	\section{符号说明}
	\begin{center}
		\begin{tabularx}{0.7\textwidth}{c@{\hspace{1pc}}|@{\hspace{2pc}}X}
			\Xhline{0.08em}
			符号 & \multicolumn{1}{c}{符号说明}\\
			\Xhline{0.05em}
			$\delta$ & 赤纬角\\
			$\beta$ & 经度\\
			$\alpha$ & 纬度\\
			$r$ & 地球半径\\
			$\gamma$ & 太阳光与杆所成的夹角\\
			$l$ & 杆的长度\\
			$l_{y}$ & 杆的影子长度\\
			$\vec{x}_{1},\vec{y}_{1},\vec{z}_{1}$ & 由杆的位置所生成的切平面的正交基\\
			$\vec{\hat{x}}_{1},\vec{\hat{y}}_{1},\vec{\hat{z}}_{1}$ & 由杆的位置所生成的切平面的单位正交基\\
			$\theta$ & 影子与北方的夹角\\
			$l_{y}(i)$ & 编号为 $i$ 的数据对应的影子长度\\
			$\theta_{i}$ & 编号为 $i$ 的数据对应的影子角度\\
			\Xhline{0.08em}
		\end{tabularx}
	\end{center}

	\section{模型的建立与求解}
	\zhlipsum*[10]
	\subsection{模型的准备}
	\zhlipsum*[11]
	\subsection{模型一的建立}
	效果见\cref{fig:1}。
	\begin{figure}[htbp]
		\centering
		\includegraphics{example-image-plain.pdf}
		\caption{content}\label{fig:1}
	\end{figure}
	\subsection{模型二的建立}
	结果见\cref{tab:1}。
	\begin{table}[htbp]
		\centering
		\caption{content}\label{tab:1}
		\begin{tabularx}{0.7\textwidth}{c@{\hspace{1pc}}|@{\hspace{2pc}}X}
		\Xhline{0.08em}
		符号 & \multicolumn{1}{c}{符号说明}\\
		\Xhline{0.05em}
		$\delta$ & 赤纬角\\
		$\beta$ & 经度\\
		$\alpha$ & 纬度\\
		$r$ & 地球半径\\
		$\gamma$ & 太阳光与杆所成的夹角\\
		$l$ & 杆的长度\\
		$l_{y}$ & 杆的影子长度\\
		$\vec{x}_{1},\vec{y}_{1},\vec{z}_{1}$ & 由杆的位置所生成的切平面的正交基\\
		$\vec{\hat{x}}_{1},\vec{\hat{y}}_{1},\vec{\hat{z}}_{1}$ & 由杆的位置所生成的切平面的单位正交基\\
		$\theta$ & 影子与北方的夹角\\
		$l_{y}(i)$ & 编号为 $i$ 的数据对应的影子长度\\
		$\theta_{i}$ & 编号为 $i$ 的数据对应的影子角度\\			\Xhline{0.08em}
		\end{tabularx}
	\end{table}
	\subsection{模型三的建立}
	\zhlipsum*[14]

	\section{模型的检验}
	\zhlipsum*[15]

	\section{模型的评价与改进}
	\zhlipsum*[16]
	\subsection{模型的优点}
	\zhlipsum*[17]
	\subsection{模型的缺点}
	\zhlipsum*[18]
	\subsection{模型的改进}
	\zhlipsum*[19]\cite{label}

	\phantomsection
	\addcontentsline{toc}{section}{参考文献}
	\begin{thebibliography}{99}
	\bibitem{label}content
	\end{thebibliography}

	\newpage
	\appendix
	\ctexset{section={
		format={\zihao{-4}\heiti\raggedright}
	}}
	\begin{center}
		\heiti\zihao{4} 附\hspace{1pc}录
	\end{center}
	\section{问题一的 MATLAB 代码}
	\begin{matlab}
clc,clear
%第七题
R71 = 1;
R72 = 2;
T7 = 1;
K7 = 1;
N7=10^5;
G71=tf(R72,R71);
G72=tf(1,[T7 1]);
G73=tf(1,[T7 0]);
G74=tf([N7*T7 0],[T7 N7]);
G75=tf([N7*K7*T7 N7*K7],[T7 N7]);
G76=tf([K7*T7 K7],[T7 0]);

subplot(2,3,1)
step(G71)
xlabel('$t$','interpreter','latex', 'FontSize', 12);
ylabel('$y$','interpreter','latex', 'FontSize', 12);
title('比例环节');

subplot(2,3,2)
step(G72)
xlabel('$t$','interpreter','latex', 'FontSize', 12);
ylabel('$y$','interpreter','latex', 'FontSize', 12);
title('惯性环节');

subplot(2,3,3)
step(G73)
xlabel('$t$','interpreter','latex', 'FontSize', 12);
ylabel('$y$','interpreter','latex', 'FontSize', 12);
title('积分环节');

subplot(2,3,4)
step(G74,10^-3)
xlabel('$t$','interpreter','latex', 'FontSize', 12);
ylabel('$y$','interpreter','latex', 'FontSize', 12);
title('微分环节');

subplot(2,3,5)
step(G75,10^-3)
xlabel('$t$','interpreter','latex', 'FontSize', 12);
ylabel('$y$','interpreter','latex', 'FontSize', 12);
title('比例微分环节');

subplot(2,3,6)
step(G76)
xlabel('$t$','interpreter','latex', 'FontSize', 12);
ylabel('$y$','interpreter','latex', 'FontSize', 12);
title('比例积分环节');
	\end{matlab}
\end{document}